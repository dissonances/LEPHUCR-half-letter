\hypertarget{saber-hacer}{%
\section{Saber hacer}}\label{saber-hacer}

\hypertarget{saber-tuxe9cnico}{%
\subsection{Saber técnico}\label{saber-tuxe9cnico}}

Wikipolítica es conocida por la imagen buenaonda que presenta
actualmente. Al buenaondismo teórico nosotras lo denominamos
\emph{ciudadanismo radical}. Recordemos que el Wikipartido, antecedente
a la Wikipolítica, era un proyecto político de naturaleza radicalmente
pirata.\footnote{Aaron Schwartz, por ejemplo.}

Uno de los problemas más graves que enfrentan las resistencias es la
articulación con otros esfuerzos. Las redes de movimientos de base,
activistas, defensores de los derechos humanos, deben colaborar con la
sociedad civil, sociedades que están un poco más regidas por lo
académico. Esto puede llegar a generar roces por la posición desde la
que nos aproximamos a los problemas. De modo que es necesario que
abracemos el dolor de quienes están en luchas más peligrosas y que
generemos un ecosistema que les dé un estatus más digno para enfrentar a
sus enemigos, que por lo general son empresas mineras extranjeras que
compran representantes mexicanos.

En este contexto, la práctica populista busca rebajar el lenguaje,
hablar al nivel del idiota porque este no es capaz de entablar un
diálogo tan rico como el del ilustrado. En este sentido, el
representante es el que salva al pueblo, intercediendo por él.

Nosotras, piratas, queremos que la \emph{masa idiota} se organice y
conforme nuevas multitudes, nuevas cuerpas políticas. Pugnamos por la
autogestión. Por eso creemos en la apertura de las herramientas.

Estamos en contra de la urbanización de la vida cotidiana y vemos esta
apertura de los saberes como una forma de crear territorios rizomáticos
comunes, siguiendo la idea deleuziana de \emph{Mil mesetas.} Nuestra
lucha cotidiana es por la autonomía.

Estamos en contra de la noción de ``conocimiento'' por considerarla un
discurso totalmente racionalista, que ignora otro tipo de saberes, como
la sensibilidad artística, la salud\footnote{Salud como buen vivir, por
  encima de la concepción ``capitalista'' clásica que define la salud en
  sentido negativo, como ausencia de enfermedad, como lo mínimo
  necesario para que las personas estén aptas para
  producir \cite{Balch2013}.} o la relación con la
tierra, la memoria y la conciencia sobre el territorio. Es decir, que
una militancia activa y consciente requiere que recuperemos la historia
como la principal fuente de testimonios y experiencias, y que estemos
conscientes de que el lugar que habitamos es una parte esencial de
nuestra identidad. Por ello, tenemos derecho a defenderla frente a los
desplazamientos forzados por megaproyectos y otras imposiciones
estatales.

Se requiere competencia técnica y claridad discursiva y representativa
para cuidar nuestro proyecto electoral y abrir el sistema de partidos.
Además de esto, también vemos un horizonte pedagógico donde una
institución tenga como objetivo la apertura de sus saberes. Que nos
enseñe crear, a escuchar y a cooperar. Así vemos la wikiescuela.

\hypertarget{saber-uxe9tico}{%
\subsection{Saber ético}\label{saber-uxe9tico}}

Reiteramos que un principio medio olvidado ha sido la empatía. Es
importante tener presente que las cosas que hacen otras personas tienen
algún sentido, al menos para ellas. En pocas palabras, no hay ninguna
forma de vida que sea intrínsecamente más valiosa que otra.

Observamos con pena que entre la banda prevalece una atmósfera de
rectitud moral, como si fuera evidente que lo que están haciendo es lo
correcto. Creemos que una de las formas de combatir esta actitud, que
lejos de permitir el diálogo, lo diluye en identidades rígidas y
alejadas entre sí, es negando cualquier intento de esencializar nuestras
luchas políticas. Después de todo, el diálogo no es más que una forma de
comunicación, y como tal, es imposible si se niega la posibilidad de
ejercerlo; por otra parte, la libertad solo existe en el momento en que
somos capaces de tomar una decisión, y la lucha política que es una
decisión, nunca es evidente.

Intuimos un desarrollo no impositivo, sino una cultura de cultivar
buenas prácticas a través de interpretaciones habituales de los
principios. Queremos resaltar que sentimos mucha cercanía axiológica con
el budismo zen, el taoísmo y el zapatismo.

Sin embargo, somos enemigas de la correctitud política que invisibiliza
las circunstancias particulares y que nos impide hablar desde lo que
sentimos y pensamos, apelando a una objetividad universal que siempre es
interpretada por algún hombre blanco, heterosexual, el único que goza
efectivamente del ideal de la ciudadanía cosmopolita.

Hay que tomar en cuenta que la autonomía y las filosofías comunes que
propugnan por la autogestión, luchan por el derecho de cada parte a
determinarse y definirse. Mientras tanto, los marxistas más dogmáticos
ven a la teoría como algo en sí misma, casi como una doctrina religiosa
de trabajo.

El dogmatismo será el enemigo en todo momento; no así la persona
dogmática. La teoría nos compromete a desarrollar prácticas y nuestras
prácticas tienen que dar siempre alguna luz respecto a lo que opinamos.
En ese sentido, también retomamos la impronta que una de nosotras
compartía sobre Hannah Arendt y el
totalitarismo \cite{Roiz2002}.

La postura provisional que por cuestiones estratégicas reconocemos como
el espíritu actual de Wikipolítica, a lo que hemos llamado
\emph{ciudadanismo radical}, consiste en abrir espacios para la sociedad
civil, asumiendo un papel como promotora del proyecto postideológico
burgués o de las clases inferiores a la dominante, para crear las
condiciones de posibilidad para debates de agenda. Es decir, debemos
abrir el gobierno a la sociedad civil para \emph{reemplazar} a la clase
burocrática, para terminar con la guerra.

\hypertarget{hacer}{%
\subsection{Hacer}\label{hacer}}

Creemos que nuestra comunicación política y el poder simbólico que ha
generado el proyecto electoral no han sido aprovechados para plantear un
programa político alcanzable bajo un \emph{ethos} distinto. Abrámonos al
sufrimiento de quienes no tienen nada. Aprendamos a ver cómo el dolor
para muchas es el motor de su voluntad para luchar y construyamos para
las pobres, para el proletariado.

No queremos gobernar, queremos que la gente pueda gobernarse maximizando
la eficiencia de sus recursos y brindándoles nuevas herramientas.
Resetear los símbolos culturales que existen alrededor de las
herramientas.

Una de las cosas que debemos tener en cuenta es que una escuela política
que impulse la colectividad debe hacer de la supresión (y gestión) del
ego una práctica cotidiana. De nuevo, en este sentido, nuestra escuela
política debe configurar un \emph{ethos.} No se trata de compartir un
canon ilustrado, sino de descubrirse como manifestaciones concretas de
la humanidad.

Wikipolítica, entendida como La Partida, puede conjugar dos herramientas
que son la escuela política (salir a la calle, al diálogo y a la guerra
de la imagen) y la reapropiación de las tecnologías para la democracia.
Nuestra Partida no debe ser más que una vía para la construcción del
movimiento de apertura tecnológica de nuestra Historia. Su compromiso es
con el paradigma interseccional, que parte de la idea de los
conocimientos situados.\cite{Haraway1988}

\hypertarget{estrategia}{%
\subsection{Estrategia}\label{estrategia}}

Necesitamos comprometernos con sentar un precedente operativo basado en
nuestro potencial tecnológico. Si lo hacemos bien, la necesidad de abrir
otras asambleas políticas en el país se esparcirá como un meme (virus
cultural), que permitirá que todas tengamos los saberes necesarios para
organizarnos en torno a lo común.

Dentro de nuestro programa político está la urgencia de abrir hasta el
último dispositivo que perpetúe, bajo la servidumbre, un grado de
opresión tal que imposibilita que una vida particular no pueda
desarrollarse como potencia. Bien claro está que la guerra contra la
Metafísica totalitaria de Occidente se encuentra en una nueva ciencia de
los dispositivos, como lo señala
Tiqqun \cite{Tiqqun2015}.

La cuestión, como la vemos, se trata de acercar el marxismo a la calle,
a la no-clase proletaria, a las indígenas, a las putas, a las trans, a
las empleadas domésticas y con mayor urgencia a las mujeres negras, cuyo
papel en la historia de este país ha sido radicalmente
invisibilizado.\footnote{Para eso están Franz
  Fanon \cite{Fanon1965} y las feministas
  afros \cite{Jabardo2008}.}

Una vez que logremos consolidar una infraestructura común y dinámicas
para comunicarnos, es necesario que abramos canales de comunicación.
Para lograrlo, podemos recurrir a estrategias muy antiguas pero
igualmente efectivas bajo una gran organización técnicamente capaz de
hacer una democracia efectiva: Periodismo popular, repartido en lugares
de resistencia a través de estructuras distribuidas por todo el país,
donde los contenidos puedan enlazar realidades locales con la nacional.

\emph{La tarea:} Crear una contraestructura que nos permita organizarnos
mejor como un todo, darle mayor poder a nuestras luchas como una
confederación de colectivos o algo así. Esta tarea no es sencilla y
descansa, en última instancia, en la confianza que exista en las otras,
todo el tiempo.

Es necesario que las clases o identidades reconocidas ellas mismas como
una subjetividad, sean capaces de oponerse materialmente a sus
opresores. Solamente es combate si el opresor no cede a la presión.

Dentro de nuestras estrategias de guerra, abogamos porque quienes
oprimen de algún modo, tengan la experiencia de la otredad en el
encuentro. Hasta ahora, solo de esa forma estamos de acuerdo con Mouffe
sobre el agonismo.

\hypertarget{tuxe1ctica}{%
\subsection{Táctica}\label{tuxe1ctica}}

Para lanzar una campaña de crecimiento dirigida al territorio
necesitamos despliegues de profesionales en Trabajo social, en
Comunicación. Por ello es vital que la wikiescuela sea capaz de
funcionar como una plataforma líquida que genere saberes comunes, pero
que también aprenda de las experiencias locales para nutrir los
protocolos operativos de las redes. Nunca bajo la intención de
reemplazar, siempre con la de proponer nuevas formas de hacer.

Convoquemos dinámicas abiertas donde estén presentes distintas formas de
vida. La Estética y el Arte tienen el potencial de ayudarnos a entablar
diálogos con diferentes grupos en una lucha común, la necesaria para
hacer de Wikipolítica ya no el Partido, sino La Partida.

Es nuestra tarea echar abajo el capitalismo, tanto el que nos habita, en
cuanto ego, como el hambre voraz de los inversionistas trajeados que
buscan extraer utilidad de cada rincón del planeta hasta dejarlo seco,
vacío, muerto. Nota: para lograrlo tenemos bien presente que las
relaciones de intercambio capitalista están sujetas a cambios que nos
obligan a estudiarlo con profundidad, tanto en su dimensión económica
como afectiva.\footnote{Brigitte Vasallo hace algo de esto en sus
  \emph{Amores, redes afectivas y
  revoluciones} \cite{Vasallo2014}.}

Pero al final, si los que se joden no pueden defenderse sin necesidad de
que un opresor tenga el cuidado de advertir que está siendo un mal amo,
entonces ahí hay un problema.

No queremos condenas a los oprimidos a buscar su liberación a través de
su enemigo. No queremos un opresor que diga ``yo debo ser quien defienda
a estas pobres mujeres''. Lo que le decimos es que si realmente es un
compañero sororo, que nos cuente todo lo que sabe sobre el enemigo y que
vuelva por donde vino. En todo caso, que se ofrezca como mártir, pero
nunca como el buen amo que, al representarnos, afianza su rol de opresor
en la Historia.
