\hypertarget{nuestro-sentir-pensar}{%
\section{Nuestro sentir-pensar}\label{nuestro-sentir-pensar}}

En la wiki, al menos dos facciones se han diferenciado por la naturaleza
de sus posicionamientos: piratas y populistas. Tal parece que existe una
rivalidad en cuanto a las prioridades de acción que siembra el
equilibrio entre las divisiones. Las cuestiones son: ¿deseamos la
revolución o deseamos un partido? ¿Qué tipo de instituciones queremos?

La tarea se torna complicada. Queremos establecer lazos de confianza
entre nosotras mientras nos preparamos para abrir masivamente el
movimiento. Para lograrlo, necesitamos analizar las impresiones de la
juventud. Una pregunta común:

\emph{¿Cómo podemos empoderar a la ciudadanía con herramientas
prácticas?}

Para nosotras, la revolución política consiste en abrir, en permitir la
autonomía de las personas en una era en la que es técnicamente posible
hacerlo. Por ello queremos abrir espacios en la wiki para crear
herramientas desde el movimiento. Esto significa reducir los costos
mediante filosofías como la de Open
Spaces \cite{Wikipedia2023b} para permitir que todas las
wikis seamos co-creadoras. De este modo, la wiki podría funcionar con
una base mínima de responsables y, al mismo tiempo, generar una cultura
que \emph{okupa} territorios en disputa.

La democracia liberal parte de asumir que somos iguales. Pero lo somos
parcialmente. Jamás podremos entender otra subjetividad, porque su mera
existencia parte de condiciones materiales e incluso espacio-temporales
únicas. Partir de ese hecho fundamental implica que tenemos un ego, un
interés individual que nos hace tener un instinto de satisfacer la
voluntad propia más que alcanzar un consenso.

Para lograr articular esta parte pirata, que okupa las estructuras de
poder, necesitamos una biblioteca multimedia común, así como fomentar
una cultura de la Inteligencia, para operar saberes comunitarios que
hagan cada vez menos necesaria la intervención de agentes armados.

Sobre el papel de la Inteligencia, nuestra labor es abrir y operar los
saberes estratégicos y tácticos, con el fin de constituirlos como un
territorio común para todo el pueblo que pertenece a nuestra nación,
pero especialmente para nosotras.

Muchas de nosotras ya fuimos integrantes del Partido Internacional de la
Juventud, pero todavía no lo sabemos. Fuimos yuppies antes de ser
tecnócratas \cite{BBC2002}. Abrimos tecnologías. Ya
hackeamos el deseo de muchos conservadores, pero la dictadura del
Espectáculo nos condujo de vuelta a las sombras.

Hay que darnos cuenta de que mucho de lo que alimenta un sentir contra
la partidocracia tiene que ver con su capacidad de hacer siempre
efectivas sus potencias. En ese sentido, es importante preguntarnos: si
estuviéramos en su lugar, ¿crearíamos otras formas de poder? ¿Por qué
desearíamos renunciar a nuestros privilegios? ¿Podemos hablar de hackeos
a la burocracia a través del deseo, de la imagen? ¿Qué papel juega la
\emph{clase} en esta lucha?

\emph{¡Nunca más una Inteligencia a la orden del patriarcado!}

La tarea puede vislumbrarse nuevamente a través de los principios. La
cuestión está en tratar de que todo lo común a nosotras (espacios,
recursos, expresiones) tengan mecanismos efectivos de implementación
local. Queremos que cualquier persona con Internet pueda acceder a
nuestros recursos y organizarse. Llegado ese momento seremos tantas
habitando la inteligencia colectiva de Wikipolítica que el problema de
cómo ir más allá de la infraestructura satelital será tan fácil como ir
a compartir con tu vecino una suerte de red local global, configurada
por una cantidad infinita de nodos interconectados, que no dependan del
poder tecnológico del Imperio.

Creemos que las instituciones deben funcionar no a través de la coerción
sino de la comprensión de la diferencia en el desarrollo de tratamientos
contra el autoritarismo. Imaginemos una apertura escalonada de
información, según lo que la gente quiera compartir y lo que la wiki
haya ganado por su mérito de honorabilidad. Programar la gestión de
responsabilidades orgánicamente. Nosotras queremos que la mediación sea
un proceso más íntimo, casi místico. Una confesión.

Nuestra visión del gobierno es comprensiva. Creemos que jugar el rol de
jefas comunitarias es para hacer una política a través de la empatía y
de la escucha. Para eso nos preguntamos constantemente ¿cómo empatizas
un problema? ¿Cómo abres los dolores o el goce de una a la otra? Como
parte de las tecnologías desde y para nosotras, pensamos que un taller
para aprender a trabajar en equipo debe incluir dinámicas para decirnos
las cosas, hablar desde nuestras experiencias de vida, confiando en que
está bien sentirse vulnerable. ¡Aquí no se vale ser
machín! \cite{SailorFag2018}

\begin{center}
    * *
\end{center}

El proletariado del que escribe Marx no puede ser entendido nunca como
una clase, porque el proletariado es la no-clase por excelencia. Son los
nadie, los que, citando a Galeano, cuestan menos que la bala que los
mata \cite{Galeano2015}. Es el niño que te vende chicles
o la embarazada que te vende cigarros afuera del antro. El proletariado
se caracteriza por no jugar ningún papel en la historia más que el de
superviviente.

Decimos que el equivalente al proletariado en la sociedad mexicana del
espectáculo es el naco. La naquez es la cualidad de ser nadie en la
cultura, de verse reducido a espectador puro, a consumidor sumiso.

\emph{En relación con el proletariado, nosotras somos la
intelligentsia.}

\emph{Somos hackers y policías del futuro.}

La ciencia o arte del control es el regalo del dominio policíaco de la
técnica. Ese es el privilegio que poseemos por ser la clase más
exquisita de todas, la clase que conoce la cultura, la
\emph{intelligentsia}. Nuestra cualidad de tecnócratas deriva de la
capacidad de operar en este dominio.

\emph{Esta potencia es circunstancia común a nosotras.}

\emph{El dominio tecnológico es poder tecnopolítico, trans-formador.}

\emph{El dominio de la técnica está formalizado en el
diseño.}\footnote{Este saber en particular plantea el problema de cómo
  aterrizar una idea a una implementación material. En este sentido, el
  diseño es una técnica de la presencia.}

\emph{Robocop dice:} ``Quiero el control del deseo para la libertad: los
policías usaremos estas tecnologías para liberarles, pero con algunas
omisiones en el proceso.''

El mismo espíritu de esta nueva forma de procesar lo real ha dado origen
a la antítesis de la cibernética y es \emph{devenir
hacker} \cite{Swartz2011}. Esta actitud ha permitido
desarrollar espacios de teorización sobre problemas
por-venir \cite{Williams2017}.
