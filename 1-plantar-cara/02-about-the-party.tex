\hypertarget{ser-o-tomar-partido}{%
\section{¿Ser o tomar Partido?}\label{ser-o-tomar-partido}}

El final del capitalismo es inminente. Nos guste o no, el capitalismo
acabará. El grado de catástrofe planetaria va a provocar un proceso de
extinción masiva en el que el sistema termine por tragarse todo.
Paralelamente a ese proceso, existe la posibilidad de que en cualquier
momento experimentemos la singularidad.

México está en guerra. Ninguna fórmula reformista ha funcionado. Debemos
pensar en una alternativa radical. Nosotras, wikis, creemos y actuamos
por una Wikipolítica que actúe en el presente y cree imágenes del
porvenir que inviten a todas a co-crear. Ahora que la gente confía en
nosotras, necesitamos construir un imaginario de ese futuro que
prometemos.

La juventud experimenta un cambio radical en la interacción con su
realidad en medio del caos. Sufrimos porque somos conscientes de que
prevalece el cinismo ilustrado en el espíritu universitario, porque
sabemos que tenemos el potencial técnico para vivir en un mundo abierto,
libre, pero no sabemos cómo crearlo. Nuestra cabeza está saturada de
imágenes, de doctrinas subliminales. Eso es el Espectáculo.

En este panorama tan desolador, compartimos con las teorías populistas
la idea de que es una lucha hegemónica. Por ello, creemos que tenemos la
responsabilidad de crear una alternativa para que las masas se percaten
de que pueden desconectarse de la Matrix y crear su propia cultura.

Debido a esto, abogamos por feminizar la hegemonía. Por eso somos La
Partida. Nuestra radicalidad se manifiesta en nuestra vocación de
máquina de guerra nómada \cite{Negri2013}. Estamos
plenamente conscientes de que la experiencia de las sombras es tan
importante para la revolución como la apertura de las instituciones
políticas y económicas.

Hoy en día estamos en condiciones de desarrollar una cibernética del
deseo, arte y ciencia del des-control de nuestras pasiones.

Des y reconfigurar aquello que mueve nuestra voluntad.

**

Ser wiki es sabernos colectivas, una inteligencia común, una cuerpa
política. Nos cuestionamos constantemente cómo vamos a reaccionar ante
la catástrofe, cómo hacemos algo en medio de esta guerra. Sentimos que
la filosofía wiki también es una práctica de la horizontalidad, se basa
en la idea de que todas aportamos y que es necesario aprender a escuchar
con empatía a otras personas.

¡Veámoslo, wikis, la tecnología es nuestra! Vivimos en un momento
histórico en el que se ha demostrado el poder de la piratería y de los
movimientos \emph{open source}, que cuestionan las formas de conocer a
través de castigos y disciplina, desarmando los dispositivos de control
del Imperio \cite{Amin2005}.

Sabemos que otras resistencias de izquierda más radical no han podido
mantener un impacto, pero sí tecnologías de lucha avanzadas. No
subestimemos sus esfuerzos. Estamos aquí porque hubo un Atenco, hubo una
\emph{otra campaña}, hubo un Syriza. Esas luchas tienen que lidiar con
el enemigo, que en todo momento busca despojarle de su terreno, de su
patrimonio natural o de su vida misma.

Ellas nos han enseñado que la autogestión no significa informalidad.
Significa comprometernos a cultivar y entrelazar los saberes necesarios
para que la gente pueda organizarse autónomamente con sus locales, con
quienes comparte una vida común. De ese modo hacemos efectivo el
principio de localismo.

Vemos una wiki que innova para nosotras. Ella organiza bancos de tiempo,
sale al Legislativo a propugnar por una economía social para buscar un
pretexto que nos ayude a encontrarnos, a organizarnos, a crear nuevos
mundos juntas.

**

Nosotras tomamos Partido. Lo escupimos.

\emph{Nosotras somos wikis, somos La Partida.}
