\hypertarget{advertencias}{%
\section{Advertencias}\label{advertencias}}

Wikipolítica actualmente es capaz de hacer frente a problemas más
complejos en lo que respecta a nuestro futuro en la(s) historia(s).

Para ganar la guerra, necesitamos ganar sensibilidades, siempre tratando
con dignidad todos los esfuerzos de otras luchas. Debemos nuestras
libertades al trabajo de otras personas que nos han precedido y que han
logrado conquistas de a poco a lo largo de la Historia.

Es un error pretender imponerse a la gente, tomar nuestra ideología como
algo evidente, con un vocabulario cerrado. A medida que aprendemos a
compartir, socializar y hacer un ejercicio mayéutico que le permita a la
banda darse cuenta por sí misma de lo que queremos hacerle ver, la lucha
será más eficaz. En este sentido, la militancia requiere aprender a
escuchar a las personas para tener una aproximación más o menos clara de
sus creencias y comprender que la persuasión se fundamenta en la
apertura misma al diálogo.

Una de nosotras nos hizo recordar la importancia de la alegría rebelde.
Gocemos que todavía estamos de pie, luchando y reconozcamos que si no
nos entendemos todavía es porque tenemos que crear un lenguaje común, no
neutralizado\footnote{Dado que neutralizar el lenguaje significa cederlo
  a la comprensión conceptual de la derecha. Mouffe sobre
  Wittgenstein: \cite{Mouffe2009}}, sino deliberado.

Disentir a través del diálogo y no de la necropolítica. Organizar
subjetividades históricas en cuerpos políticos. La ultraderecha adopta
posturas económicas de la socialdemocracia keynesiana. Si no nos
preocupamos por abrir las élites, nuestro poder se verá cada vez más
reducido. Las élites detentan el dominio material frente a los oprimidos
mediante las herramientas tecnológicas.

La técnica es el principal instrumento de dominación, los recursos
físicos se transforman a través del dominio objetivo de su uso y
distribución. El conjunto de significados que conforman el mundo de una
subjetividad dan sentido a su experiencia de vida y a su capacidad de
acatar los despliegues de poder de los dispositivos imperiales.

La experiencia fenoménica de un conjunto de significados subjetivados
constituye la realidad del mundo de esa forma-de-vida. Esta puede ser
una persona, pero ¿por qué no también un alien, o una inteligencia
artificial? La subjetividad siempre se produce en una relación
dialéctica con la otredad. Fomentar el reconocimiento de las diferencias
por encima de la construcción de pisos mínimos.

El macho consciente de su pecado diría: que se haga en mí lo que ellas
digan, que mi cuerpo soberano no sirva, sino para hacernos más libres a
todas. Solo mediante la crítica puedo intentar construir un mundo más
libre desde mi propio privilegio. Para nosotras, la crítica es un
compromiso con la destitución del policía que vive dentro de mí, es un
performance, una puesta en tela de juicio de toda palabra que sale de
mí. Toda omisión mía puede ser un acto consciente de privilegio. Solo
hacer libre a una hermana puede hacerme libre, porque en su mirada yace
mi humanidad, los arquetipos, el universo entero.

\% Used to have a page break

Nuestra tarea no es otra que unir al pasado y al futuro a través del
presente. Compartimos la filosofía oscura de anti-ilustrados como Nick
Land. Planeamos conjuros para que la gente vuelva a imaginar y hacemos
ingeniería inversa de todo dispositivo al servicio del Capital.

El macho utilitarista dirá: otro panfleto político.

Su imprudencia ignora que la puesta en escena de la presencia se muestra
en su mayor radicalidad en el texto. Hemos dado ya un paso al mostrarnos
como una práctica que hace historia en cuanto se sabe.

Entre las sombras, nos acercamos a las compañeras de Wikipolítica para
invitarlas a crear una wiki de sus saberes personales. A partir de este
momento, la impronta de este movimiento ha sido rebasada por cualquier
limitación, por cualquier argumento. A partir de ahora declaramos:

La Partida está en guerra.

Nuestras balas son imágenes.

Co-crearemos un medio.

Compañera, hoy toca mediatizarnos.

Hackear todos los dispositivos que nos impiden

imaginarnos libres.

No somos sabias filósofas que construimos con el compás y la escuadra.
Nuestro mundo se cultiva, requiere de cuidado, de escuchar la tierra, de
florecer.

La Partida es un compromiso por imaginar lo imposible, para crear
porvenires.
