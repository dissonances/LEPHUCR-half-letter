\begin{verse}
    Por dentro, la alarma
    
    suena para nosotras.
    
    La ecocatástrofe nos consume,
    
    se manifiesta
    
    en los ojos.
    
    La alarma afuera
    
    la sombra que nos cubre
    
    posibilita.
    
    Reconocer
    
    que me habita
    
    en ti también.
    
    Ser nosotras

        \hspace*{1cm}cultivar
        
        \hspace*{1cm}la palabra
        
        \hspace*{1cm}revolución.
        
\end{verse}


\hypertarget{prologue}{%
\section{Prólogo}\label{prologue}}

Esto es un (pre)texto, una provocación. Algunas aristas en nuestro
imaginario. Golpetear los horizontes personales y empezar a cultivar
visiones comunes.

\#yosoy132 fue otra de las explosiones que muestran la fragilidad e
ilegitimidad de un régimen del Estado criminal, que ha oprimido
sistemáticamente a la gente que menos tiene y que asesina a cualquier
inconforme. Wikipolítica es uno de los frutos de una generación que está
cansada de repetir fórmulas dogmáticas y deseos priístas. Somos las que
quieren atreverse a vivir, a soñar, a combatir.

La revolución que queremos es la que parte del deseo. Se trata de
recuperar la capacidad de soñar, de imaginar en un mundo en el que el
Espectáculo dicta lo que queremos, lo que necesitamos y cómo debemos
amarnos.

Las breves intervenciones de este documento no pretenden, sino crear,
destruyendo, a la par de advertir algunos fantasmas que amenazan cómo
apagar, una vez más, la llama por un cambio radical en la historia.

Todavía nos faltan 43 pero muchas más\ldots{}