\hypertarget{muchas-izquierdas-una-derecha}{%
\section{Muchas izquierdas, una
derecha}\label{muchas-izquierdas-una-derecha}}

Somos pragmáticas. No nos importa pertenecer al espectro de la izquierda
y la derecha, pues miramos a lo más urgente, una existencia material
garantizada para todas, y para lograrlo necesitamos tecnologías
divulgadas y experiencias compartidas. La Partida ha nacido con el
objetivo de luchar por esas necesidades. Nosotras queremos construir los
espacios de encuentro para que todas puedan decidir con la libertad qué
desean.

No somos de izquierda porque no queremos representar a nadie, ni
gobernar, queremos que la gente recupere su propia voz, que ella misma
pueda defender sus batallas, no queremos más paternalismos. Wikipolítica
proviene de la izquierda, sí, sentimos a las vencidas y estamos aquí
porque compartimos su dolor, porque su lucha nos habita. Compartimos la
lucha de la gente que pugna por una libertad radical, pero no queremos
ser gobierno. Queremos dar voz, pero no partimos del dolor, partimos de
la crítica como fe en la acción.

Lo que queremos es la horizontalidad del poder. Esa es nuestra causa.
Esto es un movimiento agónico que demanda consenso. El movimiento
reconoce la importancia de construir, de abrir espacios para reinventar
el sentir de la izquierda con una agenda interseccional. Esto podría
configurarse como una sociedad internacional de análisis y cooperación
que pretende incidir más allá de los medios institucionalizados.

En ese sentido, Wikipolítica sí está más allá del espectro de la
izquierda y de la derecha, lo que significa que lo que puede hacer como
un avance más allá de lo ideológico es proporcionar el aparato técnico
necesario para eficientar diferentes
luchas \cite{Tiqqun2014}.

\begin{center}
    * *
\end{center}

La izquierda también es un sentir, una gran cantidad de luchas, requiere
de herramientas como la teoría populista. En nuestra encomienda está el
desarrollo de recursos comunes para alcanzar la autonomía tecnopolítica.
La diferencia que tenemos con movimientos de izquierda es que nosotras
no tenemos esperanza.

No esperamos. Ya nos han arrebatado la idea de un futuro, lo único que
nos queda es el ahora, una práctica constante de la autonomía.
Recuperemos el valor de himnos de revolución\footnote{Parte del
  sentimiento localista nos llama a poner atención al folclor
  tradicional de cada territorio.} como el SKA, apropiémonos del perreo
y no dejemos que el Espectáculo nos diga cómo y para quién debemos mover
nuestros culos. Luchemos por combatir el amor Disney, alimentado de
fantasías y abracemos un amor de entrega, de incertidumbre, de
complicidad y de lucha \cite{Vasallo2014}.

El sentir de la izquierda es afín a la wiki. No nos representa en cuanto
parte de la misma impronta ilustrada que ignora y continúa reprimiendo
las sombras que nos habitan. Parte de nuestra filosofía del encuentro
debe propugnar por hacer frente a las sombras, las nuestras.
