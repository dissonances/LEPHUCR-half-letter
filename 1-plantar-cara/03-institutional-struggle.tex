\hypertarget{la-lucha-institucional}{%
\section{La lucha institucional}\label{la-lucha-institucional}}

Estamos en guerra. Desde 2016, por modificaciones al artículo 29
constitucional, vivimos en un estado de excepción.\cite{ElUniversal2016}
No basta con observar el despliegue de las fuerzas del orden y la
militarización para darse cuenta de que a la partidocracia, dirigida por
machos profundamente alienados e inhumanos, no le importa nada. Ni
nuestras vidas, ni las de cientos de miles que han estado enfrentados en
un conflicto bélico que sigue invisibilizado. Ellos no tienen intención
de encontrar soluciones, viven del dinero ensangrentado de las
operaciones de extracción de los recursos del país. Realmente vivimos en
la emergencia nacional.

Frente a la situación mundial, necesitamos desarrollar una política
exterior que podamos promover y difundir en la opinión pública. Abrir
líneas de estudio y establecer conexiones con los diferentes sectores
con los que estemos presentes. No necesitamos hacerlo, solamente debemos
crear las condiciones de posibilidad para que suceda. A esto nos
referimos con la idea de abrir espacios.

El porvenir al que aspiramos no es progresista. La historia no tiene un
motor, el momento político e histórico único es el ahora y la única
forma de \emph{creer} es \emph{asiendo} (aser: del juego ser-hacer).
Nosotras tenemos fe en un mundo donde quepan muchos
mundos \cite{Redaccion2023} porque peleamos por ello,
peleamos. Nuestra guerra siempre es en el presente, territorio de la
presencia.

Para llevar a cabo una transformación histórica, necesitamos ser un
espacio de encuentro entre academia, sociedad civil, empresarios,
medios, sindicatos, organizaciones de base y opinión pública.
Concentremos nuestros esfuerzos en hacer que la agenda sea un conjunto
de compromisos a largo plazo -\/-junto con nuestras líneas de acción-\/-
mientras conformamos un proyecto político bien organizado y con
plataformas tecnológicas chidas.

Creemos en una política de muchos frentes, multilateral. Desde la
izquierda y la derecha, desde arriba y desde abajo.

No olvidamos, sin embargo, que la guerra siempre ha estado. El problema
es hacer de la guerra una práctica bélica de muerte. Otra violencia, la
imagen, el arte. Crear-nos, desnudarnos, transgredirnos, son solo
algunas formas más de combatir. Necesitamos reconocer que deseamos, que
una animalidad nos habita y a partir de ello replantear.

Nosotras no queremos terminar la guerra en aras de una paz artificiosa.
Nosotras queremos que todas podamos luchar. Hay una diferencia abismal
entre la violencia total, necropolítica, y otra violencia, mucho más
vital y espontánea, que hace que un átomo se una o se separe de otro.

\begin{center}
    * *
\end{center}

La estrategia ha consistido en abrir los Congresos. Esto nos brinda
ahora la posibilidad de escribir la Historia \cite{HegelFilosofiaHistoria2023},
pues la toma del Poder Legislativo nos permitirá ser un puente de
diálogo con distintos sentires, como lo señaló Kumamoto en su campaña a
la Diputación del distrito 10 en
Zapopan \cite{Gutierrez2016}.

De manera que la toma legítima del Congreso es un puente para la
co-creación popular de un porvenir común. Esto se trata de crear
imágenes que la gente haga suyas sobre cómo activarse, cómo participar
desde su circunstancia particular en la creación de un programa común de
gobierno.

En consecuencia, la cuestión del programa de gobierno también abrirá un
debate social en torno a cuestiones éticas que hasta ahora habían sido
relegadas a los espacios privados. Podremos escapar al relativismo
posmodernista que tanto nos impide imaginar más allá de lo que el
\emph{mass media}, el Espectáculo, nos muestra.

Nuestro programa de gobierno tiene que ser un medio, estar
\emph{mediatizado.} Para lograrlo necesitamos implementar operaciones de
inteligencia sobre el tráfico de la red, entender de qué modo, bajo qué
significantes, opera el \emph{statu quo}. Es decir, cuáles son los
conceptos que están en disputa para que llevemos nuestra visión de cómo
\emph{nos} gobernamos en dirección no solo a las urnas, sino a la calle,
a la acción cotidiana.

La clave está en socializar, sistematizar y compartir nuestras
experiencias mientras luchamos.

\begin{center}
    * *
\end{center}

El problema tiene algo que ver con que la política ha sido encantada por
una magia oscura: la economía. Nuestra apuesta ética regresa a la
política, al mundo de las preguntas, a lo íntimo, a las situaciones
particulares. En ese sentido, nuestra Wikipolítica se aleja de este
mundo de luces y espectaculares al discreto para asirse de \emph{la
pregunta}, posición fundante de la crítica.

Nuestro código tiene que pensar en las instituciones a través del Diseño
social. El resultado será mecanismos que incentiven la participación más
allá de la lógica utilitarista, fundamento de la magia negra. La única
restricción serían las manifestaciones de totalitarismo, tales como la
intolerancia, el autoritarismo o la intransigencia. Los discursos que
atenten contra la libre y voluntariosa determinación de la persona deben
ser contenidos y este es un límite al que no estamos dispuestas a
renunciar.\footnote{El mejor remedio para los fascistas de sonrisa
  cínica y espíritu perverso es molerlos a palos. Pero nos conformamos
  con que sean expulsados.}

El hecho de tener como recursos de interpretación de las normas una
serie de preceptos que guían nuestra conducta y acompañan nuestras
decisiones paso a paso, potencia la apuesta ético-política de muchas de
nosotras al interior de la wiki.\footnote{Nuestra normativa se
  fundamenta en buscar las expresiones mínimas a través del retorno a la
  casuística, más allá de la pretensión universalista de los sistemas
  jurídicos legalistas que dan fundamento al Estado en la actualidad. El
  retorno a los principios y valores como guías de conducta ética son
  fundamentales para rescatar a la política, que se quedó atrapada en
  las garras de la magia negra
  contemporánea \cite{Tiqqun2013}.}

Nosotras creemos que Wikipolítica se funda en la empatía. Este
sentimiento nos permite reconocer que la diferencia radical de las
formas de vida nos obliga a reconocer que el mundo de la otra persona es
nuestro mundo, en la medida que representa la posibilidad de contemplar
el mundo desde una perspectiva única.
