\hypertarget{manifiesto-de-la-partida}{%
\section{Manifiesto de La Partida}\label{manifiesto-de-la-partida}}

\emph{``Aquel que en la guerra civil\emph{ }no tome partido será
golpeado\emph{ }por la infamia y perderá\emph{ }todo derecho político.
Solón''}

Constitución de Atenas

Hablamos desde las sombras.

Somos nuestra circunstancia y la ecocatástrofe que gira en torno a ella.
Somos burgueses ilustrados, despiertos, que quieren volver a soñar.
Somos el patriarcado, la colonia, el opresor.

En este mar de contradicciones y morales escolásticas, somos los
cínicos, amos de la técnica y policías del futuro. En nuestras manos
está siempre la posibilidad de renunciar y volver a la búsqueda de la
corona del rey.

Otra persona me salva. Su miseria me recuerda cuán cómplice soy de su
opresión, pero siento también un grato regocijo verle sufrir
\cite{Wikipedia2023}. A veces llamamos humanidad a su regocijo, otras
veces, deseo, pero ellos llaman resentimiento de clase a nuestra mirada.
La que necesitan para vivir.

\emph{¡Cuán compleja es la experiencia de acontecer!}

``Hablemos de lo imposible porque de lo posible se ha dicho demasiado'':
Un mundo donde quepan muchos mundos es ya una declaración de guerra.
Contra esta máquina devoracuerpos, contra la industria
farmacopornográfica, contra nuestras relaciones mutiladas por todas las
circunstancias que nos hacen.

``Ganemos el futuro'': Nosotros, los que ya sabemos que deseamos, los
cínicos, no queremos tal cosa como ganar nada. El fin de la guerra está
en un porvenir, imagen crítica y humilde del mundo a partir de sabernos
todos policías, todos asesinos, todos un engrane de la máquina.

Criticar es operar un distanciamiento radical sobre uno mismo. Es un
duelo del sentimiento contra el cientismo, es una campaña contra ``lo
real''. En este mundo donde me elijo al elegir mi mercancía, hemos
decidido operar la escucha, dar al otro el grado sagrado de realidad que
tienen mis convicciones, y transgredirme.

Somos piratas, hackers, chamanes, payasos. Somos los enemigos del
radicalciudadanismo, de la fantasía liberal, de invisibilizar los
géneros, las diferencias de piel, las formas-de-vida. Somos enemigos de
la civilización, del cosmopolitismo urbano.

Como en la Atenas, en esta guerra quien no toma partido es cómplice de
la desidia. Nosotros queremos cimbrar la historia, devenir real en
simulacro, queremos El Partido. Una singularidad con campos de
patatas \cite{Wikipedia2023a}.

\begin{center}
    * *
\end{center}

La naturaleza de El Partido es ser el territorio de la crítica. El
Partido, sin la antropología positiva (tatuada en nuestras vergas), no
es más que un juego.

Por este movimiento de desmodernización, de devenir residuo, El Partido,
con sus aspiraciones a conformar un nuevo Estado, es disuelto por el
ácido de su propia ambición \cite{Tiqqun2013a}. De sus
cenizas surge un ethos político al que llamamos La Partida.

La Partida es lo que quieran.

Una provocación,

el enemigo,

la policía.

Las piratas, sabias curanderas.

Incluso la Pitonisa.

O tú.

La división, el inicio.

El juego, la lucha sin mi rostro en la historia.

Una improvisación \cite{Icle2009}.

Las que tenemos miedo (de la voluntad de poder, de nuestras sombras),
las que somos jardineras silenciosas de ese porvenir, somos las que
hacemos de la crítica un compromiso, una forma de vida.

Vivimos sin futuro, sin tiempo.

La autonomía, nuestra bandera.

La técnica, aquello que hay que hacer común.

La libertad, el horizonte por el que nos cuestionamos cada paso que
damos.

LA PARTIDA, órgano de la tecnocrítica en Wikipolítica Mx
