No vivimos por sino a pesar del capitalismo \emoji{factory}, toda relación social está mediada por las mercancías \emoji{shopping_bags}. Y el capitalismo ganó hace mucho \emoji{money_bag}.

¿Cuáles son las posibilidades de actuar en en el momento más incierto \emoji{confused_face} y más absurdo \emoji{zany_face} de la Historia desde mi circunstancia específica?

Mientras cientos de activistas marchan religiosamente exigiendo al gobierno \emoji{classical_building} (como si hubiera, de hecho, algo ahí que responda al llamado) un mundo más libre \emoji{dove_of_peace}, más justo \emoji{balance_scale}, más ecológico \emoji{deciduous_tree}, los males propios de nuestro tiempo están presentes en todo momento en nuestra vida. Se trata de una configuración tecnológica de la realidad \emoji{computer}, una disposición de las cosas para que la gente actúe y responda frente a ellas de cierto modo. En la era de información, el rostro del totalitarismo se confunde en un virus capitalista \emoji{face_with_medical_mask} que ciertamente ha dominado la nuda vida: en la sociedad no hay vida así sin más. Corporaciones \emoji{office_building}, máquinas \emoji{robot} y aparatos ejercen su poder sin ninguna resistencia organizada capaz de proponer una alternativa universal al estado actual de las cosas.

Vivimos una imperceptible guerra civil \emoji{globe_with_meridians} en la que el enemigo a vencer no está ni siquiera en el montón de idiotas que rigen las corporaciones y los gobiernos nacionales \emoji{school} sino en una religiosidad \emoji{folded_hands} que profesa simpatía por la dominación mercantil \emoji{shopping_cart}.
