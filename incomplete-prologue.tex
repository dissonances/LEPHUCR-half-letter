No vivimos por, sino a pesar del capitalismo \faicon{industry}, toda relación social está mediada por las mercancías \faicon{shopping-bag}. Y el capitalismo ganó hace mucho \faicon{money}.

¿Cuáles son las posibilidades de actuar en el momento más incierto y más absurdo de la Historia desde mi circunstancia específica?

Mientras cientos de activistas marchan religiosamente exigiendo al gobierno \faicon{building-o} (como si hubiera, de hecho, algo ahí que responda al llamado) un mundo más libre \faicon{flag-o}, más justo \faicon{balance-scale}, más ecológico \faicon{tree}, los males propios de nuestro tiempo están presentes en todo momento en nuestra vida. Se trata de una configuración tecnológica de la realidad \faicon{laptop}, una disposición de las cosas para que la gente actúe y responda frente a ellas de cierto modo. En la era de información, el rostro del totalitarismo se confunde en un virus capitalista \faicon{bug} que ciertamente ha dominado la nuda vida: en la sociedad no hay vida así sin más. Corporaciones \faicon{building}, máquinas \faicon{gears} y aparatos ejercen su poder sin ninguna resistencia organizada capaz de proponer una alternativa universal al estado actual de las cosas.

Vivimos una imperceptible guerra civil \faicon{globe} en la que el enemigo a vencer no está ni siquiera en el montón de idiotas que rigen las corporaciones y los gobiernos nacionales, \faicon{university} sino en una religiosidad \faicon{dollar} que profesa simpatía por la dominación mercantil \faicon{shopping-cart}.
